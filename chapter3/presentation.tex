\documentclass[
    11pt, % Set the default font size, options include: 8pt, 9pt, 10pt, 11pt, 12pt, 14pt, 17pt, 20pt
    %
    aspectratio=169, % Uncomment to set the aspect ratio to a 16:9 ratio which matches the aspect ratio of 1080p and 4K screens and projectors
]{beamer}
\usepackage{hyperref}
%\graphicspath{{Images/}{./}} % Specifies where to look for included images (trailing slash required)
\usepackage{booktabs} % Allows the use of \toprule, \midrule and \bottomrule for better rules in tables

%\usepackage{appendixnumberbeamer} %If you want a separate slide counter for your appendix

%%% Customize Theme %%%%%%%%%%%%%%%%%%%%%%
\usetheme{Madrid} % You can use other themes too, but this changes many things. I've found Madrid to be the best for this color scheme

%fg = font color
%bg = background color

% ! WARNING ! : Many colors are linked to multiple attributes, so changing one color can have unexpected changes!

% If you want to tweak the shading of orange and red, tweak the below 2 lines:t
\definecolor{myRed}{RGB}{120,4,4}
\definecolor{myOrange}{RGB}{227, 125, 0}

% Bottom right hand color
\setbeamercolor*{structure}{bg=myRed!20,fg=myRed!90}

\setbeamercolor*{palette primary}{use=structure,fg=white,bg=structure.fg} %?
\setbeamercolor*{palette secondary}{use=structure,fg=myRed,bg=white}
    %bottom left of footer & bar between title & top bubbles
\setbeamercolor*{palette tertiary}{use=structure,fg=white,bg=myRed} 

\setbeamercolor{frametitle}{bg=myRed!85,fg=white} %title of each slide

\setbeamercolor*{titlelike}{parent=palette primary} %?
%\setbeamercolor{titlelike}{parent=palette primary,fg=structure.fg!50!myRed}

%for miniframe (very top) AND center footer
\setbeamercolor{section in head/foot}{fg=myOrange, bg=white}

%%% Specific Colors %%%
\setbeamercolor{item projected}{bg=myOrange}
\setbeamertemplate{enumerate items}{bg=myOrange}

\setbeamercolor{itemize item}{fg=myOrange}
\setbeamercolor{itemize subitem}{fg=myOrange}

\setbeamercolor{button}{bg=myOrange}

%%% Edits ONLY the TOC slide %%%
\setbeamercolor{section in toc}{fg=black}
\setbeamercolor{subsection in toc}{fg=black}

%%% Block Colors %%%
% Standard block %
    \setbeamercolor{block title}{bg=myOrange, fg=white}
    \setbeamercolor{block body}{bg=myOrange!20}

% Alerted block % If you want to customize it's color
    %\setbeamercolor{block title alerted}{bg=cyan, fg=white}
    %\setbeamercolor{block body alerted}{bg=cyan!10}

% Example block % If you want to customize it's color
    %\setbeamercolor{block title example}{bg=cyan, fg=white}
    %\setbeamercolor{block body example}{bg=cyan!10}

%---------------------------------------------------------
%	SELECT FONT THEME & FONTS
%---------------------------------------------------------
\usefonttheme{default} % Typeset using the default sans serif font
\usepackage{palatino} % Use the Palatino font for serif text
\usepackage[default]{opensans} % Use the Open Sans font for sans serif text
\useinnertheme{circles}

%---------------------------------------------------------
%	SELECT OUTER THEME
%---------------------------------------------------------
% Outer themes change the overall layout of slides, such as: header and footer lines, sidebars and slide titles. Uncomment each theme in turn to see what changes it makes to your presentation.

%\useoutertheme{default}
%
\useoutertheme{miniframes}

%\useoutertheme{infolines}
%\useoutertheme{smoothbars}
%\useoutertheme{sidebar}
%\useoutertheme{split}
%\useoutertheme{shadow}
%\useoutertheme{tree}
%\useoutertheme{smoothtree}

%---------------------------------------------------------
%	PRESENTATION INFORMATION
%---------------------------------------------------------

\title[Gödel, Turing and Friends]{Quantum Computing Since Democritus}
\subtitle{Scott Aaronson}
\author[Quantum Computing Since Democritus]{Chapter 3: Gödel, Turing and Friends}

\institute[]{Saumya Chaturvedi \\ \smallskip }
\date[24 March 2023]
%\date[\today]


%---------------------------------------------------------
%---------------------------------------------------------
%---------------------------------------------------------
\begin{document}

%---------------------------------------------------------
%	TITLE SLIDE
%---------------------------------------------------------
\section{}
\begin{frame}
	\titlepage % Output the title slide, automatically created using the text entered in the PRESENTATION INFORMATION block above
 
\end{frame}

%---------------------------------------------------------
%	TABLE OF CONTENTS SLIDE
%---------------------------------------------------------

\begin{frame}
	\frametitle{Table of Contents} % Slide title, remove this command for no title
	
	\tableofcontents % Output the table of contents (all sections on one slide)
	%\tableofcontents[pausesections] % Output the table of contents (break sections up across separate slides)
\end{frame}

%---------------------------------------------------------
%	PRESENTATION BODY SLIDES
\section{Gödel's Completeness Theorem}

\begin{frame}{Gödel's Completeness Theorem}
\begin{itemize}
    \item First Order Logic is all you need.
    \item If, starting from some set of axioms, you can’t derive a contradiction using these rules, then the axioms must have a model.
    \item Everything from Fermat's Last Theorem can be proved by applying these rules over and over again.
    
\end{itemize}
    
\end{frame}
\begin{frame}{Proof}
    \begin{itemize}
        \item “extracting semantics from syntax.”
        \item We cook up objects to order as the axioms request them!
        \item If inconsistency found, it suggests inconsistency in the original axioms.
    \end{itemize}
\end{frame}
\begin{frame}{Application}
    \begin{itemize}
        \item Löwenheim-Skolem Theorem: If a countable theory has a model, it has a countable model.
        \item Every consistent set of axioms has a model of at most countable cardinality.
        \item Because you can only cook up objects to order a countably infinite number of times!
    \end{itemize}
\end{frame}


\section{Gödel's Incompleteness Theorem}
\begin{frame}{Gödel's (In)completeness Theorem}
\begin{itemize}
    \item For a computable (finite axioms), consistent (no contradictions) set of axioms, there's a true statement within those axioms which cannot be proven from those axioms.
    \item Proof: 30 pages vs 2 lines.
\end{itemize}
    
\end{frame}
\begin{frame}{Proof Method 1: Encoding}
\begin{itemize}
    \item Similar to Scott's friend's method of realizing arrays.
    \item Very cool demonstration on whiteboard.
\end{itemize}
    
\end{frame}


\section{Turing and the Halting Problem}

\begin{frame}{Turing Machine and Assumptions}
 \begin{itemize}
    \item In 1936, "computer" meant a woman.
    \item Realising a computer machine which does the following:
    \begin{itemize}
        \item Writes calculations on a square paper, each symbol per square.
        \item Reads 1 symbol at a time, and is able to go back and forth on the tape.
        
        \newline
    \end{itemize}
    
    \item How does it make instantaneous decisions?
    \begin{itemize}
        \item Depends on the symbol currently being read,
        \item and the machine’s current “internal configuration”
or “state.” 
        \newline
    
    \end{itemize}
    \item What should the machine do?
    \begin{enumerate}
        \item  Write a new symbol in the current square,
overwriting whatever symbol is there,  
        \item  Move backward or forward one square, and
        \item  Switch to a new state or halt.
        \newline
    
    \end{enumerate}
    \item The number of possible internal states should be finite.
\end{itemize}
\end{frame}

\begin{frame}{Implications}
    \begin{itemize}
        \item Universal programmable computers can exist.
        \item Inventing the Halting Problem
        \item Hard problem, solving this solves other problems like Goldbach's Conjecture
    \end{itemize}
\end{frame}
\begin{frame}{Halting Problem and Proof}
 \begin{itemize}
        \item Given a program, can we determine if it halts ever?
        \item Proof by contradiction that a program P to solve Turing problem exists.
        \item Modify P to produce a new program P’ that:
        \begin{itemize}
            \item  runs forever if Q halts given its own code as input, or
            \item halts if Q runs forever given its own code as input. 
        \end{itemize}
        \item Feed P' its own code as input.
        \item P' will run forever if it halts, and halt if it runs forever. So it doesn't exist.
    \end{itemize}
    
\end{frame}
\begin{frame}{Halting Proving Incompleteness}
    \begin{itemize}
        \item Suppose: A consistent, computable proof system F proving/disproving any statement about the integers.
        \item<1-> Halting Problem: about integers, find proof within F.
        \item<2-> But program to solve Halting Problem can't exist.
        \item<3-> F can't exist.
    \end{itemize}
\end{frame}


\section{Second Incompleteness Theorem}
\begin{frame}{Second Incompleteness Theorem}
   \begin{itemize}
    \item Back to program P' fed to itself, why not use this program as proof?
    \item Hidden assumption behind it: F is consistent.
    \item If F was inconsistent, it could prove P halting even if it ran forever.
    \item But if F could prove its consistency, it could also prove P halting or running forever.
    \item Only possible conclusion: F is consistent but can't prove its own consistency.
    \item To prove consistency of powerful theories, we need even more powerful theories. (Illustration on whiteboard)
\end{itemize} 
\end{frame}


\begin{frame}{Implications }
\begin{itemize}
    \item Finding bigger infinities (large Cardinals) to prove consistencies. Example: PA and ZF.
    \item<1-> A quick question to test your understanding: while we can’t prove in PA that Con(PA), can we least prove in PA that Con(PA) implies Con(ZF)?
    \item<2-> No! Because then we could prove in ZF that Con(PA) implies Con(ZF).
\end{itemize}
\end{frame}

\section{Completeness vs Incompleteness}
                      
\begin{frame}{Peano Arithmetic's example}
\subtitle{How do these 2 not cancel each other out?}
\begin{itemize}
    \item “self-hating theory”: PA + the assertion of its own inconsistency. (whiteboard)
    \item Axioms keep cooking up large fictitious numbers to create a model for theories.
    \item The point of the Completeness Theorem is that the whole infinite set of fictitious numbers the axioms cook up will constitute a model for PA – just not from ordinary positive integers!
    \item For the latter, we move to Incompleteness theorem.
\end{itemize}
\end{frame}

\section{Just Believe in Yourself!}

\begin{frame}{Just Believe in Yourself!}
\begin{itemize}
    \item Puzzle from last chapter
    \item Let's assume we can prove the Riemann Hypothesis!
    \item $\zeta$(s) = 1 + $\frac{1}{2^s}$ + 1/3s + 1/4s + ...
    \item Löb's Theorem: a mathematical system cannot assert its own soundness without becoming inconsistent. 
\end{itemize}   
\end{frame}


\section{Independence}

\begin{frame}{How does AC and CH fit into this?}
\begin{itemize}
    \item Is continuum (cardinality of the real numbers) true or false?
    \item Godel 1939: assuming the Axiom of Choice (AC) or the Continuum Hypothesis (CH) can never lead to an inconsistency.
    \item Can we also assume AC and CH are false?
    \item Paul Cohen 1963: CH is not provable from the axioms of set theory. ("forcing")
    \item The independence of AC and CH from ZF set theory is itself a theorem of Peano Arithmetic.
    \item Do we ever really talk about the continuum, or do we only ever talk about finite sequences of symbols that talk about the continuum?
\end{itemize}
    
\end{frame}

\section{Aha! Connection}
\begin{frame}{Physically Meaningless}
\begin{itemize}
    \item How many different positions of a pen on paper? $\aleph_1$?  $2^{\aleph_0}$?
    \item A physically meaningless question, but requires a physical theory!
    \item Quantum mechnanics is somehow "quantized" yet continuous.
    \item Scott's assumption: \href{https://www.google.com/search?q=hilbert+space&rlz=1C1CHBD_enIN1013IN1013&sxsrf=AJOqlzWw92tsrRiWqy1BJLIvsWLRpFqF-Q:1679559886629&source=lnms&tbm=isch&sa=X&ved=2ahUKEwjfpK-T0PH9AhUpzzgGHZL6DwUQ_AUoAXoECAEQAw&biw=1280&bih=577&dpr=1.5#imgrc=b_3ncQ7CReTCqM}{Hilbert space}(the space of all possible quantum states of some system) is finite-dimensional.
    \item There are continuous parameters (probabilities or amplitudes) but not directly observable, we're "shielded" from them. (waveforms collapsing on measurement.)
\end{itemize}
\end{frame}

\section{Exercises and Further Reading}
\begin{frame}{Exercises and Further Reading}
    \centering
    %\caption{Hawkins et al, 2015}
    \includegraphics[angle=0, width=12cm, height=6.65cm]{exerciseAndReadings.png}
\end{frame}


\end{document}

%---------------------------------------------------------


